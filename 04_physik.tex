\section{Theoretische Grundlagen}
\label{sec:physik}
In diesem Abschnitt werden die theoretischen Grundlagen aufgeführt, die für die Projektbearbeitung notwendig sind. Es wird hierbei hauptsächlich auf das Thema der Dosierung konzentriert, da diese derzeit den Hauptschwerpunkt des Projektes darstellt.

\subsection{Dosierung mittels Tropftrichter oder Tropf}
Die einfachste Variante um eine automatische Dosierung über einen festgelegten Zeitraum zu dosieren, ist ein höher gelegener, gefüllter Behälter mit einer Leitung zum Reaktionsraum. Typische Beispiele sind aus dem chemischen Labor der Tropftrichter und aus der Medizin eine Infusion. 

\subsubsection*{Tropftrichter}
Ein Tropftrichter beschreibt ein Glasgerät, welches im Labor zum Zutropfen von Flüssigkeiten und Lösungen zu einer Reaktionsmischung verwendet. Hierbei lässt sich über einen Hahn eine genaue Dosierung realisieren \cite{Hunig.2006}.
Vorteil des Tropftrichter ist seine leichte Handhabung und die meist vorhandene Schliffverbindung, welche ein einfaches Montieren ermöglicht.
Nachteil des Tropftrichters ist jedoch, dass sich mit ihm keine genauen Volumenströme, außer über die Tropfgeschwindigkeit einstellen lassen können. Zudem verringert sich die Geschwindigkeit mit Verringerung des hydrostatischen Druckes durch den Füllstand im Tropftrichter (siehe \hyperref[sec:torricelli]{\ref{sec:torricelli} \textsc{Torricelli}-Theorem}). Ebenfalls ein Nachteil ist, dass sich hochviskose Stoffe kaum bis gar nicht auf diese Art und Weise zu dosieren lassen.

\subsubsection*{Schwerkraftinfusion}
Alternativ findet sich im medizinischem Alltag ebenfalls eine Dosierung mit kleinen Flussraten in Form einer Schwerkraftinfusion wieder. Diese funktioniert, ebenso wie der Tropftrichter, aufgrund des hydrostatischen Druckes und kann üblicherweise Flussraten von \SI{5}{\milli \liter \per \hour} bis \SI{300}{\milli \liter \per \hour} erreichen \cite{pfm_medical}.
Vorteil einer Schwerkraftinfusion zeigt sich in ihrem Aufbau, da über eine sogenannte Tropfenkammer und einen Durchflussregler eine genaue Tropfgeschwindigkeit einstellbar ist. Für den Anwender lassen sich so, einfacher als beim Tropftrichter die Tropfgeschwindigkeit und damit auch der Volumenstrom regulieren.
Die Nachteile der Schwerkraftinfusion gleichen sich mit denen des Tropftrichters. Es wäre demnach eine regelmäßige Kontrolle der Tropfgeschwindigkeit notwendig aufgrund der sich ändernden Fließeigenschaften mit der Zeit. Somit wären regelmäßige Kontrollen der Fließrate innerhalb von 20-\SI{30}{\minute} nötig \cite{OnlinePortalfurprofessionellPflegende.06.03.2017}.

\subsection{Ausflussgeschwindigkeit - \textsc{Torricelli-Theorem}}
\label{sec:torricelli}
Sowohl Tropftrichter als auch 

\subsection{Dosierung mittels Pumpen}
\subsubsection*{Zahnradpumpe}
\subsubsection*{Spritzenpumpe}
\subsubsection*{Schlauch-Peristaltik-Pumpe}
\subsubsection*{Membranmagnetpumpe}

\subsection{Temperaturprofile mittels Thermostat}
\anmerkung{Anleitungen verlinken bzw. Arbeit in externes Dokument aufteilen}

\subsection{Rührer und Drehzahl}
Ankerrührer hochviskose Medien
Drehzahl nicht genau bestimmbar