%Dokumentklasse

%draft als optionohne bilder für bessere performance
%\documentclass[a4paper,12pt,]{scrreprt}

%normal mit Bildern
\documentclass[a4paper, 11pt, draft=false]{scrartcl}

\usepackage[left= 3cm,right = 2.5cm, bottom = 3cm,top = 2.5cm]{geometry}


% ============= Packages =============
% Dokumentinformationen
\usepackage[
pdftitle={Projektbericht Thermische Verfahrenstechnik},
pdfauthor={Roman-Luca Zank},
hidelinks %Entfernt Rahmen von Links
]{hyperref}

% Standard Packages
\usepackage[utf8]{inputenc}
\usepackage[ngerman]{babel}
\usepackage[none]{hyphenat}
\hyphenpenalty=5000
\tolerance=5000
\providecommand\phantomsection{}

%Tabellen Packages
\usepackage{makecell}
\usepackage{colortbl}
\usepackage{longtable}
\usepackage{tabularx}
\usepackage{tabulary}
\usepackage{multirow} 

%Grafik Packages
\usepackage{graphicx}
\graphicspath{{img/}}
\usepackage{fancyhdr}
\usepackage{lmodern}
\usepackage[table]{xcolor}
\usepackage{adjustbox}

%Diagramm Packages
\usepackage{tikz}
\usetikzlibrary{patterns}
\usepackage{pgfplots}
\pgfplotsset{/pgf/number format/use comma}
\pgfplotsset{grid style={white!90!black}}

%Chemie Packages
\usepackage{mhchem}
\usepackage{chemfig}

%Andere Packages
\usepackage{adjustbox}
\usepackage{placeins}
\usepackage{booktabs}
\usepackage{caption}
\usepackage[list=true]{subcaption}
\usepackage[shortlabels]{enumitem}
\usepackage{menukeys}

% Mathe Packages
\usepackage{mathtools}
\usepackage{amsfonts}
\usepackage{amsmath}
%Automatisch cdot statt *
\DeclareMathSymbol{*}{\mathbin}{symbols}{"01}
%nur letzte Zeile der Gleichung nummerieren
\makeatletter
\def\Let@{\def\\{\notag\math@cr}}
\makeatother

%Abkürzungsverzeichnis
\usepackage{acronym}

%Seiten Packages
\usepackage{lastpage}
\usepackage{pdfpages}

%Einheitenpackage
\usepackage{siunitx}  
\sisetup{locale = DE, 
	per-mode=fraction,
	inter-unit-product=\ensuremath{\cdot},
	detect-weight = true,
	quotient-mode=fraction
}
%neue Einheiten definieren
\DeclareSIUnit\xyz{xyz}	
\DeclareSIUnit\rpm{rpm}	
\DeclareSIUnit\mws{mWS}	
\DeclareSIUnit\degrees{^\circ}
\DeclareSIUnit\kmeter{\raiseto{3}\meter}	


%========== angepasste Comands ================
%angepasster \today Command
\newcommand{\leadingzero}[1]{\ifnum #1<10 0\the#1\else\the#1\fi}
\newcommand{\todayDE}{\leadingzero{\day}.\leadingzero{\month}.\the\year}

%Tippbox
\newcommand{\tipp}[2]{{
		\textit{#1:}\\
		\vspace*{-5mm}
		\fbox{\parbox{\linewidth}{#2}}}}

%Uhr
\newcommand{\uhr}[2]{#1:#2 Uhr}

%Bild
\newcommand{\bild}[3]{
	\begin{figure}[h!]
		\centering
		\includegraphics[width=#3\textwidth]{img/#2}
		\caption{#1}
		\label{fig:#2}
	\end{figure}
	\FloatBarrier
	\vspace*{-10mm}
}

%Bild in MInipage
\newcommand{\minibild}[3]{
	%\vspace{#3}
	\centering
	\includegraphics[width=#3\textwidth]{img/#2}
	\caption{#1}
	\label{fig:#2}
}

%Anmerkung
\newcommand{\anmerkung}[1]{
	\textcolor{red}{#1}
}

%nicht einrücken nach Absatz
\setlength{\parindent}{0pt}

%Abb. statt Abbildung
\addto\captionsngerman{%
	\renewcommand{\figurename}{Abb.}%
	\renewcommand{\tablename}{Tab.}%
}

\urlstyle{same}

% ============= Kopf- und Fußzeile =============
\pagestyle{fancy}
%
\lhead{}
\chead{}
\rhead{}%\slshape }%\leftmark}
%%
\lfoot{}
\cfoot{}
\rfoot[{\thepage\ of \pageref*{LastPage}}]{Seite \thepage\ von \pageref*{LastPage}}
%%
\renewcommand{\headrulewidth}{0pt}
\renewcommand{\footrulewidth}{0pt}

%Fußnotelinie
%\let\footnoterule

%Fußnote mit Klammer
\renewcommand*{\thefootnote}{(\arabic{footnote})}

% ============= Dokumentbeginn =============

\begin{document}
%Seiten ohne Kopf- und Fußzeile sowie Seitenzahl
\pagestyle{empty}

%\begin{center}
\begin{tabular}{p{\textwidth}}


\begin{center}
\includegraphics[scale=0.75]{logos.jpg}\\
\end{center}

\begin{center}
	\LARGE{\textsc{Hochschule Merseburg}}
\end{center}

\vspace*{12.5mm}

\rule{\textwidth}{0.4pt}
\begin{center}
\textbf{\LARGE{Temperierung und Dosierung eines Laborreaktors}}
\end{center}
\vspace*{-5mm}
\rule{\textwidth}{0.4pt}

\vspace*{12.5mm}

\begin{center}
	\Large{\textsc{
			Projektbericht \\
			Thermische Verfahrenstechnik II\\
	}}
\end{center}

\vspace*{12.5mm}

\begin{center}
\Large{\textbf{vorgelegt von:}} \\ 
\end{center}
\begin{center}
\large{Roman-Luca Zank} \\
\end{center}


\vspace*{12.5mm}

\begin{center}
\begin{tabular}{lll}
\large{\textbf{Betreuung:}}&& \large{Herr Ramhold}\\
&&\\
\large{\textbf{Versuchsdurchführung:}}&& \large{Ende Mai bis Anfang Juni}\\
&&\\
\large{\textbf{Abgabe:}}&& \large{\todayDE}\\
\end{tabular}
\end{center}

\end{tabular}
\end{center}
\vfill
\large{Merseburg den \todayDE}


%\include{14_danksagungen}

%\include{15_zusammenfassung}

% Beendet eine Seite und erzwingt auf den nachfolgenden Seiten die Ausgabe aller Gleitobjekte (z.B. Abbildungen), die bislang definiert, aber noch nicht ausgegeben wurden. Dieser Befehl fügt, falls nötig, eine leere Seite ein, sodaß die nächste Seite nach den Gleitobjekten eine ungerade Seitennummer hat. 
%\cleardoubleoddpage

% Pagestyle für Titelblatt leer
\pagestyle{empty}

%Seite zählen ab
\setcounter{page}{0}

%Titelblatt
\begin{center}
\begin{tabular}{p{\textwidth}}


\begin{center}
\includegraphics[scale=0.75]{logos.jpg}\\
\end{center}

\begin{center}
	\LARGE{\textsc{Hochschule Merseburg}}
\end{center}

\vspace*{12.5mm}

\rule{\textwidth}{0.4pt}
\begin{center}
\textbf{\LARGE{Temperierung und Dosierung eines Laborreaktors}}
\end{center}
\vspace*{-5mm}
\rule{\textwidth}{0.4pt}

\vspace*{12.5mm}

\begin{center}
	\Large{\textsc{
			Projektbericht \\
			Thermische Verfahrenstechnik II\\
	}}
\end{center}

\vspace*{12.5mm}

\begin{center}
\Large{\textbf{vorgelegt von:}} \\ 
\end{center}
\begin{center}
\large{Roman-Luca Zank} \\
\end{center}


\vspace*{12.5mm}

\begin{center}
\begin{tabular}{lll}
\large{\textbf{Betreuung:}}&& \large{Herr Ramhold}\\
&&\\
\large{\textbf{Versuchsdurchführung:}}&& \large{Ende Mai bis Anfang Juni}\\
&&\\
\large{\textbf{Abgabe:}}&& \large{\todayDE}\\
\end{tabular}
\end{center}

\end{tabular}
\end{center}
\vfill
\large{Merseburg den \todayDE}
 %Protokolle
%\include{01_titel2} %Seminar-/Abschlussarbeit

% Pagestyle für Rest des Dokuments
\pagestyle{fancy}

%Inhaltsverzeichnis
\tableofcontents
\thispagestyle{empty}

%Inhalt

%Verzeichnis aller Bilder
\label{sec:bilder}
\listoffigures
\addcontentsline{toc}{section}{Abbildungsverzeichnis}
\thispagestyle{empty}

%Verzeichnis aller Tabellen
\label{sec:tabellen}
\listoftables
\addcontentsline{toc}{section}{Tabellenverzeichnis}
\thispagestyle{empty}

\newpage
%\section*{Symbolverzeichnis}
\label{sec:symbole}
% Table generated by Excel2LaTeX from sheet 'Daten'
\begin{table}[h!]
	\renewcommand*{\arraystretch}{1.2}
	\centering
	%\rowcolors{2}{white}{gray!25}
	\begin{tabulary}{1.0\textwidth}{C|p{.7\textwidth}|C}
			\hline
			\textbf{Symbol} & \textbf{Beschreibung} & \textbf{Einheit}\\
			\hline
			$\alpha$ & Wärmeübergangskoeffizient & \si{\watt\per\meter \per \kelvin}\\
			\hline		
	\end{tabulary}
\end{table}%
\FloatBarrier

\newpage

\section{Einleitung und Versuchsziel}
\label{sec:aufgabenstellung}
%In der Aufgabenstellung wird (in eigenen Worten und ganzen Sätzen) formuliert, was das Ziel des 
%Versuches ist.  
%[Beachten Sie die eigentliche Aufgabenstellung in den Versuchsanleitungen sowie die Hinweise zur Auswertung!] 

Im folgenden Versuch wird die Konzentration an Stickstoffdioxid in der Raumluft bzw. unter einem Abzug des Labors Hg/E/2/17 bestimmt. Da Stickstoffdioxid normalerweise nur in geringen Mengen emittiert wird, trifft man in diesem Versuch eine theoretische Annahme. Diese umfasst, dass im Labor beispielsweise ein Druckgefäß geplatzt ist oder der Abzug nicht ordnungsgemäß arbeitet und deshalb die Konzentration an \ce{NO2} in der Raumluft bestimmt werden muss.\\
Arbeitsmethodisch wird eine Langzeitbeprobung durchgeführt, um das \ce{NO2} anzureichern. Infolgedessen wird mittels externen Standards und UV-VIS-Spektroskopie die Konzentration an \ce{NO2} bestimmt.



\section{Theoretische Grundlagen}
\label{sec:physik}

\subsection{Dosierung mittels Pumpen}
\subsubsection*{Spritzenpumpe}
\subsubsection*{Zahnradpumpe}
\subsubsection*{Schlauch-Peristaltik-Pumpe}
\subsubsection*{Membranmagnetpumpe}

\subsection{Dosierung mittels Tropftrichter oder Tropf}
\subsubsection{medizinischer Tropf}
\subsubsection{Tropftrichter}

\subsection{Temperaturprofile mittels Thermostat}
\anmerkung{Anleitungen verlinken}
\section{Geräte und Chemikalien}
\label{sec:geraete}

\textbf{Geräte:}
\begin{itemize}
	\begin{minipage}{0.49\textwidth}
		\item portable Pumpe: \\
		\textsc{Ametek} - Alpha 1 Airsampler
		\item Spektralphotometer: \\
		\textsc{Analytik Jena} - Spekol 1500
		\item Kunststoff-Pipetten
		\item \SI{50}{\milli \meter} Küvetten
		\item Stoppuhr
	\end{minipage}
	\hfill
	\begin{minipage}{0.49\textwidth}
		\item Schraubverschlüsse mit Dichtungen für Waschflaschen
		\item Kunststoffschläuche
		\item Messkolben verschiedener Volumina
		\item Mess- und Vollpipetten verschiedener Volumina
		\item \SI{100}{\milli \liter} Waschflaschen mit Frittenensatz
	\end{minipage}
\end{itemize}
\vspace*{5mm}
\textbf{Proben/Chemikalien:}
\begin{itemize}
		\item \textsc{Saltzmann}-Reagenz
		\item destilliertes Wasser
		\item Vergleichslösung mit Natriumnitrit (\SI{1}{\milli \liter} enthält umgerechnet \SI{1}{\micro \gram} \ce{NO2})
\end{itemize}





\section{Versuchsdurchführung}
\label{sec:durchfuerung}

\subsection*{Probenahme der Raumluft}
Der Versuch begann um 8:02 Uhr mit der 90-minütigen Probenahme von \ce{NO2} unter einem Abzug im Labor Hg/E/2/17. 
Ziel ist es mit Hilfe des Versuchsaufbaus \ce{NO2} aus der Raumluft in \SI{25}{\milli \liter} \textsc{Saltzmann}-Lösung zu absorbieren \mbox{(siehe Abb. \ref{fig:versuchsaufbau})}.
Hierfür wurde nach Aufbau des Versuchsstandes die Pumpe eingeschaltet. Der Volumenstrom wurde am Ende der gesamten Versuchsdurchführung bestimmt.

\bild{Versuchsaufbau-Probenahme}{versuchsaufbau}{0.75}

\subsection*{Kalibrierung mit Natriumnitrit-Lösung}
Während die Probenahme lief, wurden parallel dazu die Kalibrierlösungen hergestellt. Hierfür wurde eine Vergleichslösung mit einer Massenkonzentration von \SI{1,5}{\milli \gram \per \liter} Natriumnitrit zur Verfügung gestellt. Umgerechnet hatte die Lösung eine Massenkonzentration von \SI{1}{\milli \gram \per \liter} \ce{NO2}. In Tabelle \ref{tab:kalibrierlosungen} sind die Verdünnungsreihen nach der Versuchsanleitung mit den benötigten Volumina dargestellt.
\vspace*{-5mm}
\begin{table}[h!]
	\renewcommand*{\arraystretch}{1.2}
	\centering
	\rowcolors{2}{gray!25}{white}
	\caption{Kalibrierlösungen}
	\label{tab:kalibrierlosungen}
	\resizebox{\textwidth}{!}{
		\begin{tabulary}{1.2\textwidth}{C|C|CC}
			\hline
			\textbf{Kalibrierlösung} & \textbf{Zielkonzentration} $\left[\si{\milli \gram \per \liter}\right]$ & \textbf{Volumen Vergleichslösung} $\left[\si{\milli \liter}\right]$& \textbf{Volumen \textsc{Saltzmann}-Lösung} $\left[\si{\milli \liter}\right]$\\
			\hline
			K1 & 0,01 & 0,5 & 49,5\\
			K2 & 0,02 & 1,0 & 49,0\\
			K3 & 0,03 & 1,5 & 48,5\\
			K4 & 0,04 & 2,0 & 48,0\\
			K5 & 0,06 & 3,0 & 47,0\\
			K6 & 0,08 & 4,0 & 46,0\\	
			\hline			
	\end{tabulary}}
\end{table}%
\FloatBarrier
Je höher die Konzentration der Kalibrierlösung gewesen war, desto intensiver erschien die Farbe des Farbstoffes.
Nach dem Herstellen der Lösungen und 15-minütigem Warten wurden mit den Kalibrierlösungen K2, K4 und K5 die Wellenlänge des Absorbtionsmaximums $\lambda_{max}$ bestimmt. Zunächst ist dafür eine Küvette mit destilliertem Wasser als Referenz im Spektralphotometer vermessen und hinterlegt worden. Danach erfolgte die Vermessung der genannten Kalibrierlösungen und deren Messwerte für $\lambda_{max}$ wurden arithmetisch gemittelt. Die Ergebnisse dieser Messungen finden sich unter Abschnitt \ref{sec:ergebnisse}.

Nach der Bestimmung der Wellenlänge des Absorbtionsmaximums $\lambda_{max}$ konnten nun die Absorbanzen für alle Kalibrierlösungen bei dieser Wellenlänge bestimmt werden. Mit Hilfe dieser Absorbanzen ist nun ein Aufstellen der Kalibriergerade zur Messung der Konzentration der Raumluftprobe möglich. Mehr dazu unter Abschnitt \ref{sec:ergebnisse}.

\subsection*{Messung der Raumluftprobe:}
Sobald die 90 Minuten vergangen waren, wurde die Probenlösung nochmals für 15 Minuten stehen gelassen, sodass sich der Farbstoff vollständig ausbilden konnte. Währenddessen wurde die Probenahmeapparatur abgebaut und die Pumpe an den Seifenblasenzähler angeschlossen.
Die Messungen der Absorbanz der Raumluftprobe erfolgte nach Ablauf der Wartezeit auf gleiche Weise wie die Messung der Kalibrierlösungen. Es wurden drei Messungen von Absorbanzen bei der ermittelten Wellenlänge $\lambda_{max}=\SI{548}{\nano\meter}$ für die Raumluftprobe durchgeführt.

\subsection*{Volumenstrom der Pumpe}
Der Volumenstrom der Pumpe wurde mittels Seifenblasenzähler ermittelt. Eine Skizze des Versuchsaufbaus ist in Abbildung \ref{fig:seifenblase} zu sehen. Hierfür wurde die Pumpe mit der oberen Schlauchtülle des Seifenblasenzählers verbunden und eingeschaltet. Am unteren Ende des Seifenblasenzählers wurde die Seifenblasenlösungen an die Öffnung gegeben, sodass diese von der Pumpe angesaugt wurde. Es bildeten sich flache Seifenblasen, welche sich entlang der Skalierung bis zum Doppelboden des Seifenblasenzähler bewegten. Nach dem mehrere Blasen das obere Ende des Zählers erreicht hatten, wurde mit der Messung des Volumenstroms begonnen.\\
Hierfür wurde erneut eine Seifenblase durch ein Ansaugen der Pumpe im Seifenblasenzähler gebildet. 

Sobald diese die beginnende Skalierung für den \SI{500}{\milli \liter}-Abschnitt des Zählers erreichte, wurde die Zeit gemessen die die Seifenblase brauchte, um die obere Marke von \SI{500}{\milli \liter} zu erreichen. Insgesamt wurde diese Messung dreimal durchgeführt und eine mittlere Zeit berechnet, die Seifenblasen benötigten. Aus diesem Wert wird unter \mbox{Abschnitt \ref{sec:ergebnisse}} der Volumenstrom der Pumpe bestimmt.

\bild{Skizze Seifenblasenzähler}{seifenblase}{0.33}


\newpage
\section{Ergebnisse}
\label{sec:ergebnisse}


\newpage
\section{Diskussion der Ergebnisse}
\label{sec:diskussion}

\anmerkung{Kleinst nötiger Volumenstrom liegt bei 6,78 ml/h über 5 h !}

\section{Zusammenfassung und Fazit}
\label{sec:zusammenfassung}


%\input{011_fehlerbetrachtung}

\include{08_literatur}

%\section*{Anhang}
\addcontentsline{toc}{section}{Anhang}
\label{sec:anhang}

\anmerkung{Bilder Versuchsaufbau}

%\include{09_erklaerung}

\end{document}
