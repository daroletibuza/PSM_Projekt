\section{Zusammenfassung und Fazit}
\label{sec:zusammenfassung}
Die Zielstellung die vereinfachte Prozessführung aus Tabelle \ref{tab:verinfachteAnforderungen} ist nicht vollständig gelungen. Hauptproblem des vereinfachten Prozesses stellt die Dosierung dar. Über die langen Zeiträume über die die jeweiligen Feeds zu dosiert werden sollen, kommen für eine konstante Dosierung lediglich Pumpensysteme in Frage. Die für den Versuch genutzten Pumpen sind hierbei nicht für die derzeitige Ausführung des Prozesses geeignet oder nur unter ständiger Nachjustierung zu Lasten einer erschwerten Reproduzierbarkeit einsetzbar.\\
Die Einstellung der Temperaturprofile für den vereinfachten Prozess erwies sich als erfolgreich, jedoch sollte in der weiteren Bearbeitung darauf geachtet werden, dass nicht die Temperatur des Reaktormantels, sondern die der Reaktionsmischung im Fokus steht. Demnach ist zu erwarten, dass sich Aufheiz- und Kühlprozesse in der realen Ausführung der Prozesse 1 und 2 leicht verzögern werden. Ein Anpassung dieses Aspektes in der Programmierung des Thermostates ist jedoch leicht möglich.\\

Fazit des Versuches ist, dass bevor weitere Anforderung der Prozesse wie Stickstoffatmosphäre oder Wasserdampfdestillation überprüft werden, die Dosierung der Feeds umzusetzen ist. Hierfür sollten Möglichkeiten anderer Pumpen, wie zum Beispiel Spritzen- oder Schlauchpumpen, sowie eine verzweigte Führung der Volumenströme in Betracht gezogen werden. Die Temperierung des Reaktors wird als erfolgreich bewertet.
